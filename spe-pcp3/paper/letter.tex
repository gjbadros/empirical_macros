% $Id$
% Doc: /scratch/gjb/spe-pcp3/paper/letter.tex
% Original Author:     Greg Badros <gjb@>
% Created:             Wed Jul 14 1999
%

\documentclass{letter}
%\usepackage{fullpage}
\usepackage{personal}
\address{Dept.\ of Computer Science \& Engineering \\
University of Washington  \\ 
Box 352350 \\
Seattle, WA 98195-2350 \\
(206) 616-3997}

\newcommand{\pcp}{\mbox{\textsf{PCp}$^3$}}

\signature{Greg J.\ Badros}
\begin{document}
\begin{letter}{
Software Practice and Experience}

\opening{Dear Editor:}

Enclosed please find our newest version of "A Framework for
Preprocessor-Aware C Source Code Analyses" by Badros and Notkin, which we
have revised based on the comments of the two referees.

Most of our changes fall into two categories: (1) high-level changes
motivated by the referees' comments about a lack of clarity in our
goals, our approach, etc.; and (2) a relatively large set of detailed
changes based on specific low-level concerns of the referees.  Here we
briefly recount the most important issues in these categories.

High-level concerns:

\begin{itemize}

\item Referee A had few high-level concerns; the largest was a nagging
sense that we didn't do a thorough job of solving the general problem.  That
is, indeed, an accurate appraisal of our results.  On the whole, solving
general problems requires even deeper insight than we have gained from our
design and implementation, largely because we intentionally chose a narrow
domain, aiding in the analysis of C code written using the preprocessor.  To
address the broader questions would require much more effort and experience
with (at least) several more implementations.  We believe that our approach
is attractive because of its effectiveness in the narrow domain and because
it seems relatively straightforward to apply to other, similar domains.  But
we are still a far cry from a general solution, as desirable as one is.

\item Referee B had similar concerns about the applicability of the
approach.  We have tried to clarify our results in this regard (not only in
this letter but also in the paper itself).

\end{itemize}

Specific issues:

\begin{itemize}

\item Referee B is concerned with our terminology being confusing (with
respect to terms like "preprocessed", etc.).  Unfortunately, there is no
standard terminology in the field for these things.  So we have chosen one
terminology, defined what we used, and have used it consistently throughout
the paper.  (In earlier drafts, virtually every reader had a different view
on the "right" ways to name things; we have experience when we claim
that a standard naming scheme is unavailable.)

\item Although "\pcp" may not be an ideal name, it is indeed what we call
it: "pee-see-pee-cubed".  We have noted our preferred pronunciation in
the paper to aid the reader when re-encountering the name.

\item We have added a brief discussion of the selection of Perl for
handling hooks.  To summarize, we view the cost in the learning curve for C
programmers with no Perl experience to be less than the cost of continuing
with C and incurring a significantly longer compile-run-debug cycle.

\item We have added the arguments to the hooks reference in Appendix A 
      to be more explicit about what can be expressed with our hooks,
      and what information each hook receives from the framework.

\end{itemize}

Overall, we have worked to improve the paper based on the referees'
comments, and we hope it is now acceptable for publication in SPE.

\closing{Best regards,}
\end{letter}
\end{document}

% Local Variables: 
% mode: latex
% TeX-master: t
% TeX-master: t
% End: 
