\begin{description}
%% See ccd_lexical_category in em_analyze for the routines
%% which implements these heuristics

\item[Package-specific] These guards check conditions specific
      to the given package.  They do not fit any of the other categories.

\item[Portability, feature] These guards check symbols which
      describe specific capabilities of the target machine or operating
      system (e.g., \texttt{BYTEORDER}, \texttt{BROKEN_TIOCGWINSZ}).

\item[Portability, language or library] These guards check
      symbols that compilers may predefine, that standard libraries may
      define, or that the package may define as part of the build
      process to indicate existence of compiler, language, or library
      features (e.g., \texttt{GNUC}, \texttt{STDC}, or \texttt{HAS_BOOL}).

\item[Portability, machine] These guards check symbols which
      name the operating system or machine hardware (e.g.,
      \texttt{sun386} or \texttt{MACINTOSH}.
      
\item[Portability, system macro] These guards check symbols
      which are commonly defined constants or pseudo-inline functions in
      system or language libraries (e.g., \texttt{O_CREATE},
      \texttt{isalnum}, or \texttt{S_IRWXUSR}).

\item[Miscellaneous system] These guards test reserved symbols
      (i.e., those beginning with two underscores) and which do not
      fit any other category.
      
\item[Debugging] These guards control inclusion of debugging
      or tracing code.  Macro names tested include the
      substring \texttt{DEBUG} or \texttt{TRACE} (or both).
      
\item[Commenting] These guards either definitely succeed and
      have no effect as currently written (e.g., \texttt{#ifdef 1}), or
      definitely fail and unconditionally skip a block (e.g.,
      \texttt{#ifdef 0}).
      
\item[MI prevention] These guards encompass an entire file to
      ensure that the enclosed code is seen only once per translation
      unit by the compiler.  Such guards are indicated by a trailing
      \texttt{_H} or \texttt{_INCLUDED} in the macro name they check.

\item[Redefinition suppression] These guards are immediately
      followed a definition of the same symbol, thus avoid preprocessor
      warnings about a redefinition of a name (e.g., \texttt{#ifndef
      FOO} followed by \texttt{#define FOO ...} and \texttt{#endif} on
      subsequent lines).

\item[Mixed categories] These guards test multiple symbols
      which independently fall into different categories (e.g.,
      \texttt{#if defined(STDIO_H) || SYSV_SIGNALS}).

\end{description}
