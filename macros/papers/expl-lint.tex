
\begin{description}
\item[free variables]
  See page~\pageref{page:freevar} for a discussion of issues related to
  free variables.  We specifically check for side-effected formal arguments
  as well; see below.

\item[multiple formal uses]
        Some argument is used as an expression multiple times, so any side
        effects in the actual argument expression will occur multiple
        times.  
        Given a macro defined as
\begin{verbatim}
    #define EXP_CHAR(s) (s == '$' || s == '`' || s == CTLESC)
\end{verbatim}
        an invocation such as {\tt \verb|EXP_CHAR|(*p++)} increments the
        pointer by three locations rather than just one as intended (and as
        would occur were \verb|EXP_CHAR| a function).  Even if the argument
        has no side effects, as in \verb|EXP_CHAR|(peekc(stdin)), repeated
        evaluation may be unnecessarily expensive.
        
        Some C dialects provide an extension for declaring a local variable
        within an expression.  In GNU C~\cite{GCC}, this is achieved in the
        following manner:
\begin{verbatim}
    #define EXP_CHAR(s) ({ int _s = (s); (_s == '$' || _s == '`' || _s == CTLESC) })
\end{verbatim}

\item[unparenthesized formal uses]
        Some argument is used as a subexpression (i.e., is adjacent to an
        operator) without being enclosed in parentheses, so that precedence
        rules could result in an unanticipated computation being performed.
        For instance, in
\begin{alltt}
    #define DOUBLE(i) (2*i)
    \ldots\ DOUBLE(3+4) \ldots
\end{alltt}
        the macro invocation computes the value 10, not 14.
        This warning is suppressed when the argument is the entire body
        or is the last element of a comma-delimited list (which has 
        low precedence).

\item[unparenthesized body]
        The macro body is an expression which ought to be parenthesized to
        avoid precedence problems at the point of use.  For instance, in
\begin{alltt}
    #define DOUBLE(i) i+i
    \ldots\ 3*DOUBLE(4) \ldots
\end{alltt}
        the expression's value is 16 rather than 24.
        
        This warning is applicable only to macros that expand to an
        expression and is suppressed if the body is a single token or a
        function call (which has high precedence).

\item[doesn't swallow semicolon]
        The macro body takes arguments and expands into a statement or
        multiple statments.  Thus, its invocations look like function
        calls, but it cannot be legally used like a function call, as in
\begin{alltt}
    #define ABORT() kill(getpid(),SIGABRT);
    \ldots
    if (*p == 0)
      ABORT();
    else \ldots
\end{alltt}
        because {\tt ABORT();} expands to two statements (the second a null
        statement), which is unsyntactic between the {\tt if} condition and
        {\tt else}.

        Macros without arguments, such as {\tt \#define \verb|FORCE_TEXT|
        \verb|text_section|();}, suppress this warning on the theory that their
        odd syntax will remind the programmer not to add the usual semicolon.

        The solution to this problem is to wrap the macro body in
\begin{alltt}
             do \verb|{| \ldots\ \verb|}| while (0)
\end{alltt}
        which is a partial statement that requires a final semicolon.  To
        our surprise, we found few uses of this standard, widely-recommended
        construct but many error-prone uses of statement macros followed by
        semicolons.

\item[null body with arguments]
        The macro is a null define of the form {\tt \#define name(e)} 
        which might have been intended to be {\tt \#define name (e)}.
        An empty comment is the idiomatic technique for indicating that the
        null definition is not a programming error, so a comment where the macro
        body would be suppresses this error, as in
\begin{verbatim}
    #define __attribute__(Spec) /* empty */
    #define ReleaseProc16(cbp) /* */
    #define __inline /* No inline functions.  */
    #define inline /**/
\end{verbatim}

\item[side-effected formal]
        A formal argument is side-effected.  This is erroneous if the
        argument is not an lvalue.  A similar constraint applies to
        reference parameters in C++, which can model such macro arguments.

\item[inconsistent arity]
        The macro name is defined multiple times with different arity; for example,
\begin{verbatim}
    #define ISFUNC 0
    #define ISFUNC(s, o) ((s[o + 1] == '(')  && (s[o + 2] == ')'))
\end{verbatim}
        This may indicate either a genuine bug or a macro name used for
        different purposes in different parts of a package, in which case
        the programmer must take care that the two are never simultaneously
        active (lest one override the other).  The latter situation may be
        caught by Cpp's redefinition warnings, if the macro name is not
        subjected to {\tt \#undef} before the second definition.

\item[swallows else]
        The macro, which ends with an {\tt else}-less {\tt if} statement,
        swallows any {\tt else} clause that follows it.  For instance, after
\begin{verbatim}
    #define TAINT_ENV() if (tainting) taint_env()
    #define merge_(a,b) if (TM_DEFINED (b)) (a) = (b);
\end{verbatim}
        a use like
\begin{alltt}
    if ({\em condition})
      TAINT_ENV();
    else \ldots
\end{alltt}
        results in the {\tt else} clause being executed not if  the
        condition is false, but if it is true (and {\tt tainting} is also
        true).
        
        This problem results from a potentially incompete statement (though
        an {\tt if} statement doesn't require an {\tt else} clause) which
        may be attached to some following infomation.  It is the mirror of
        the ``doesn't swallow semicolon'' problem listed above which
        resulted from a too-complete statement which failed to be
        associated with a textually subsequent token.  The solution is
        similar: either add an else clause with an empty statement, as in
        {\tt \#define ASSERT(p) if (!(p)) botch(\verb|__STRING|(p)); else}, or
        wrap statements in {\tt \verb|{| \ldots\ \verb|}|} and wrap partial
        statements in {\tt do \verb|{| {\rm \ldots}\ \verb|}| while (0)}.

\item[bad formal name]
        The formal name is not a valid identifier or is a reserved word
        (possibly in another dialect of C).
\begin{verbatim}
#define CR_FASTER(new, cur) (((new) + 1) < ((cur) - (new)))
\end{verbatim}
        This presents no difficulty to Cpp, but a programmer reading the
        body (especially a more complicated one) may become confused.

\end{description}



%%% Local Variables: 
%%% mode: latex-mode
%%% TeX-master: "emp-use-2"
%%% End: 
