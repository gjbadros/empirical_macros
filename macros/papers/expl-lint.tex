
\begin{description}
\item[inconsistent arity]
        the macro is defined multiple times with different arity.
        This (and "all warnings by name") is reported as a percentage of
        names; other numbers are reported as a percentage of definitions.

\item[null body with args]
        The macro is of the form
\begin{verbatim}
                #define name(e)
\end{verbatim}
        which might have been intended to be
\begin{verbatim}
                #define name (e)
\end{verbatim}
        This warning is suppressed if a comment appears where the macro
        body would be, as in
\begin{verbatim}
                #define name(e)         /* Do nothing in the FOO case */
\end{verbatim}

\item[unparenthesized body]
        The macro body is an expression which ought to be parenthesized.
        This warning is suppressed if the body is a single token or a
        function call (which has high precedence)

\item[doesn't swallow semicolon]
        The macro body takes arguments and expands into a statement or
        multiple statments.  Thus, its uses look like function calls, but
        it cannot be legally used where a function call can be, as in
\begin{verbatim}
             if (*p != 0)
               MACRO_CALL (p, lim);
             else ...
\end{verbatim}
        because the presence of two statements -- the expansion of the macro and
        a null statement -- between the `if' condition and the `else'
        is unsyntactic.  The solution is to wrap the macro body in
\begin{verbatim}
             do {...} while (0);
\end{verbatim}

\item[bad formal name]
        The formal name is not a valid identifier or is a reserved word in
        some dialects of C (eg, {\tt new}).  CPP will work fine, but a
        programmer reading the body may become confused.

\item[multiple formal uses]
        Some argument is used as an expression multiple times, so any side
        effects in the expression will occur multiple times.

\item[unparenthesized formal uses]
        Some argument is used as a subexpression (ie, adjacent to an
        operator) without being enclosed in parentheses, so that precedence
        rules could result in an unanticipated computation being performed.
        For instance, in
\begin{verbatim}
                #define DOUBLE(i) (2*i)
                DOUBLE(3+4)
\end{verbatim}
        the macro body computes the value 10, not 14.
        This warning is suppressed when the argument is the entire body
        or is the last element of a comma-delimited list.

\item[side-effected formal]
        A formal argument is side-effected.  This is erroneous if the
        argument is not an lvalue.
\end{description}



%%% Local Variables: 
%%% mode: plain-tex
%%% TeX-master: "emp-use-2"
%%% End: 
