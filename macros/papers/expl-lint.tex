LINT:

inconsistent arity
        the macro is defined multiple times with different arity.
        This (and "all warnings by name") is reported as a percentage of
        names; other numbers are reported as a percentage of definitions.

null body with args
        The macro is of the form
                #define name(e)
        which might have been intended to be
                #define name (e)
        This warning is suppressed if a comment appears where the macro
        body would be, as in
                #define name(e)         /* Do nothing in the FOO case */

unparenthesized body
        The macro body is an expression which ought to be parenthesized.
        This warning is suppressed if the body is a single token or a
        function call (which has high precedence)

doesn't swallow semicolon
        The macro body takes arguments and expands into a statement or
        multiple statments.  Thus, its uses look like function calls, but
        it cannot be legally used where a function call can be, as in
             if (*p != 0)
               MACRO_CALL (p, lim);
             else ...
        because the presence of two statements -- the expansion of the macro and
        a null statement -- between the `if' condition and the `else'
        is unsyntactic.  The solution is to wrap the macro body in
             do {...} while (0);

bad formal name
        The formal name is not a valid identifier or is a reserved word in
        some dialects of C (eg, "new").  CPP will work fine, but a
        programmer reading the body may become confused.

multiple formal uses
        Some argument is used as an expression multiple times, so any side
        effects in the expression will occur multiple times.

unparenthesized formal uses
        Some argument is used as a subexpression without being enclosed in
        parentheses, so that precedence rules could result in an
        unanticipated computation being performed.  For instance, in
                #define DOUBLE(i) (2*i)
                DOUBLE(3+4)
        the macro body computes the value 10, not 14.
        This warning is suppressed when the argument is the entire body
        or is the last element of a comma-delimited list.

side-effected formal
        A formal argument is side-effected.  This is erroneous if the
        argument is not an lvalue.

===========================================================================

Assignment operators have potentially unexpected results.  A macro argument
that is assigned to is similar to a pass-by-reference function argument and
need only be noted in the macro's documentation.  A macro that assigns a
global variable also presents no difficulties in understanding or
translation into a C++ inline function.  Assignment to a local variable
that is free in the macro body, however, demands that such a variable exist
wherever the macro is invoked, and assigns to different variables at
different invocations.\footnote{By contrast, LCLint considers assignment to
  a macro argument dangerous but does not appear to check for assignments
  to local variables.~\cite{Evans:LCLint}} Such a macro implements a
restricted form of dynamic scoping by capturing the instance of a variable
visible at the point of macro invocation.

