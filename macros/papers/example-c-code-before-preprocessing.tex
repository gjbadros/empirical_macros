
%// FROM gs262/gxfill.c
\begin{verbatim}
#include "gxcpath.h"

/* Import the fixed * fixed / fixed routine from gxdraw.c. */
/* The second argument must be less than or equal to the third; */
/* all must be non-negative, and the last must be non-zero. */
extern fixed fixed_mult_quo(P3(fixed, fixed, fixed));

/* Define the structure for keeping track of active lines. */
typedef struct active_line_s active_line;
struct active_line_s {
	gs_fixed_point start;		/* x,y where line starts */
	gs_fixed_point end;		/* x,y where line ends */
	fixed dx;			/* end.x - start.x */
#define al_dx(alp) ((alp)->dx)
#define al_dy(alp) ((alp)->end.y - (alp)->start.y)
	fixed y_fast_max;		/* can do x_at_y in fixed point */
					/* if y <= y_fast_max */
#define set_al_points(alp, startp, endp)\
  (alp)->dx = (endp).x - (startp).x,\
  (alp)->y_fast_max = max_fixed /\
    (((alp)->dx >= 0 ? (alp)->dx : -(alp)->dx) | 1) + (startp).y,\
  (alp)->start = startp, (alp)->end = endp
#define al_x_at_y(alp, yv)\
  ((yv) == (alp)->end.y ? (alp)->end.x :\
   ((yv) <= (alp)->y_fast_max ?\
    ((yv) - (alp)->start.y) * al_dx(alp) / al_dy(alp) :\
    (stat(n_slow_x),\
     fixed_mult_quo(al_dx(alp), (yv) - (alp)->start.y, al_dy(alp)))) +\
   (alp)->start.x)
	fixed x_current;		/* current x position */
	fixed x_next;			/* x position at end of band */
	const segment *pseg;		/* endpoint of this line */
	int direction;			/* direction of line segment */
#define dir_up 1
#define dir_down (-1)
/* "Pending" lines (not reached in the Y ordering yet) use next and prev */
/* to order lines by increasing starting Y.  "Active" lines (being scanned) */
/* use next and prev to order lines by increasing current X, or if the */
/* current Xs are equal, by increasing final X. */
	active_line *prev, *next;
/* Link together active_lines allocated individually */
	active_line *alloc_next;
};

/* Define the ordering criterion for active lines. */
/* The xc argument is a copy of lp2->x_current. */
#define x_precedes(lp1, lp2, xc)\
  (lp1->x_current < xc || lp1->x_current == xc &&\
   (lp1->start.x > lp2->start.x || lp1->end.x < lp2->end.x))

#ifdef DEBUG
/* Internal procedures for printing active lines */
private void
print_active_line(char *label, active_line *alp)
{	dprintf5("[f]%s %lx(%d): x_current=%f x_next=%f\n",
	         label, (ulong)alp, alp->direction,
	         fixed2float(alp->x_current), fixed2float(alp->x_next));
	dprintf5("    start=(%f,%f) pt_end=%lx(%f,%f)\n",
	         fixed2float(alp->start.x), fixed2float(alp->start.y),
	         (ulong)alp->pseg,
	         fixed2float(alp->end.x), fixed2float(alp->end.y));
	dprintf2("    prev=%lx next=%lx\n",
		 (ulong)alp->prev, (ulong)alp->next);
}
private void
print_line_list(active_line *flp)
{	active_line *lp;
	for ( lp = flp; lp != 0; lp = lp->next )
	   {	fixed xc = lp->x_current, xn = lp->x_next;
		dprintf3("[f]%lx(%d): x_current/next=%g",
		         (ulong)lp, lp->direction,
		         fixed2float(xc));
		if ( xn != xc ) dprintf1("/%g", fixed2float(xn));
		dputc('\n');
	   }
}
\end{verbatim}
