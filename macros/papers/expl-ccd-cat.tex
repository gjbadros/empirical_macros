The following categories are based on macro names:

\begin{description}
%% See ccd_lexical_category in em_analyze for the routines
%% which implements these heuristics

\item[Package-specific] 
  These symbols are specific to the given package.  They do not fit any of
  the other categories.

\item[Portability, machine]
  These symbols name the operating system or machine
  hardware (e.g., \texttt{sun386} or \texttt{MACINTOSH}).
      
\item[Portability, feature] These symbols describe specific parameters
      or capabilities of the target machine or operating system (e.g.,
      \texttt{BYTEORDER}, \verb|BROKEN_TIOCGWINSZ|).  
      
%      These symbols are different from ``portability, machine'' because
%      they may correspond to multiple machines or architectures.

\item[Portability, system macro]
  These symbols are commonly defined constants or
  pseudo-inline functions in system or language libraries (e.g.,
  \verb|O_CREATE|, \texttt{isalnum}, or \verb|S_IRWXUSR|).

\item[Portability, language or library]
  These symbols are predefined by a compiler, defined by a standard
  library, or defined by the package as part of the build
  process to indicate existence of compiler, language, or library features
  (e.g., \texttt{GNUC}, \texttt{STDC}, or \verb|HAS_BOOL|).

\item[Miscellaneous system]
  These symbols are reserved (i.e., they begin with two underscores) and do
  not fit any other category.
      
\item[Debugging]
  These symbols control inclusion of debugging or tracing code.  The macro
  names include \texttt{DEBUG} or \texttt{TRACE} (or both).
      
\item[Multiple inclusion prevention]
  These guards encompass an entire file to ensure that the enclosed code is
  seen only once per translation unit by the compiler.  Such guards are
  indicated by convention with a trailing \verb|_H| or \verb|_INCLUDED| in the macro name
  they check.
\end{description}


The below three categories consider the entire guard or the context of
the directive.  These categories have precedence over the above name-based
heuristics:

\begin{description}
\item[Commenting] These guards either definitely succeed and
  have no effect as written (e.g., \texttt{\#ifdef 1}), or definitely fail
  and unconditionally skip a block (e.g., {\tt \#ifdef (0 \&\&
    \verb|OTHER_TEST|)}).  These guards are used to comment out code or to
  override other conditions (e.g., to unconditionally enable a previously
  experimental feature).
      
\item[Redefinition suppression] These guards test non-definedness of
  symbol, and control only  a definition of the same symbol, thus avoid preprocessor
      warnings about a redefinition of a name (e.g., \texttt{\#ifndef
      FOO} followed by \texttt{\#define FOO ...} and \texttt{\#endif}).
    
    The purpose is to provide a default value used unless another part of
    the system, or the compilation command, specifies another value.

\item[Mixed categories] These guards test multiple symbols
      which independently fall into different categories (e.g.,
      {\tt \#if defined(\verb|STDIO_H|) || \verb|SYSV_SIGNALS|}).

\end{description}

%%% Local Variables: 
%%% mode: latex
%%% TeX-master: "emp-use-2"
%%% End: 
