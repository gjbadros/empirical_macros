% -*- Mode: Tex -*-
% for LaTeX
\documentclass[11pt]{letter}
\usepackage{fullpage}

\begin{document}

\signature{\mbox{Michael Ernst, Greg Badros, and David Notkin}}


\begin{letter}{Jens Palsberg \\
    1398 Computer Science Building \\
    West Lafayette, Indiana, 47907-1398}

\opening{Dear Jens,}

At long last, here is a major revision of TSE \#105066, ``An empirical
analysis of C preprocessor use,'' by Ernst, Badros, and Notkin.  We are
grateful for the detailed, thoughtful, and very extensive comments by the
reviewers.  We apologize for the long delay in revising the paper, but the
new version has been substantially improved to account for those comments,
including developing a new framework to support the analyses, re-analyzing
the data using this framework, and adding new analyses.  Every part of the
paper has been rewritten for clarity, style, and to present additional
information.  We feel that this major modification is an improvement over
the original submission; we hope that the improvements further satisfy the
reviewers about the publishable quality of the paper.

Because the reviewer comments ran to 16 single-spaced pages, we will not
attempt to discuss each one in detail here.  Instead, we focus in this note
on what we believe are the most critical and crucial aspects of the
reviews.

\begin{enumerate}

\item
  Most notably, we have performed some research that was originally
  relegated to future work.  Our free variable analysis correctly
  distinguishes between between macros that refer to global variables and
  macros that refer to local variables or parameters.  We also accurately
  distinguish macro uses from references to other symbols (such as
  variables) with the same name.  While our results are not substantially
  altered by this change, we could not have known that in advance and it
  was also valuable to improve precision in these respects.

\item
  Another change to our analysis is that we only analyze one hardware and
  operating system configuration rather than all such configurations.  This
  reduces the amount of code we analyze and the number of multiple
  definitions of macros, especially system-dependent ones.  This makes our
  analysis even more compelling, for it removes a potential criticism (not
  raised by any of the reviewers) that our numbers are inflated by system
  dependences.

\item
  We more precisely explain the exact process we followed, both in the
  methodology description of section 3 and elsewhere in the paper as
  appropriate.  (For instance, the descriptions of macro dependences are
  more extensive and formal.)  This should address the reviewers' concerns
  about what data we gathered, how we gathered it, and why readers should
  believe we've done it properly.

\item
  We have rewritten the paper to avoid the use of words such as ``innocuous''
  and ``problematic'', and to justify why certain practices are likely, or
  unlikely, to present difficulties to program understanding by humans or
  by tools.  (Some uses of these words remain, changed to ``potentially
  problematic uses'', indicating that this issue is not cut-and-dried.)

\item
  Reviewers 1 and 3 were also concerned with how we determined the taxonomy
  that we present.  We have rewritten substantial pieces of the paper to
  clarify this.  Additionally, when we introduce our taxonomies, we have
  added more examples to ensure that at least one example illustrates each
  taxon.

\item
  We have also substantially expanded the related work to fully summarize
  the taxonomies and suggestions of previous work.  This makes it easier to
  compare our taxonomy, which is generally finer than theirs but not always
  perfectly comparable.  We decided that a tabular presentation of the
  comparison was not particularly illuminating.

\item
  We also expanded other aspects of related work.  (However, we were still
  unable to find any work in the software complexity metrics literature
  about the C preprocessor.)

\item
  We have revised every figure in the paper based on specific comments of
  the reviewers.  Several figures have been removed on the advice of the
  reviewers.  We have used more standard log-scale axes in two cases, and we
  have ensured that each caption gives examples of how to read or interpret
  the figure.  (The labels for the x axis of Figure 14 still need to be
  moved to the left, so they mark boundaries rather than bars; we will
  correct this in the copy-editing phase.  However, it is not possible to
  omit or merge the two left bars, as reviewer 3 suggested, without losing
  information.)

\item
  The reviewers suggested a number of additional avenues to explore;
  examples included comparisons to FlexLint/PC-Lint, correlating to version
  numbers, bug rates, etc.  Given the length of the paper (the current
  version is somewhat larger than the originally submitted version, to
  accommodate the collective suggestions of the reviewers) and given the
  extensive work (not only in rewriting) required by the revision, we chose
  not to extend the scope of the paper further.  Although these suggestions
  (and others we have developed on our own and through discussions with
  others) are excellent, they can reasonably be deferred to later research.
  As Reviewer 1 noted: ``Maybe the authors can convince colleagues to
  perform a similar study for a different set of source texts, and compare
  the results.''  Perhaps one such study could extend the results in these
  suggested dimensions.

\item
  Reviewer 3 said: ``You've taken a lot of measurements, but I didn't learn
  much.  What are readers to *do* differently, or to *understand* better as
  a result of your analysis?  What conclusions do you want us to draw from
  the data?  Are there other conclusions that we might have hoped to draw
  but that are not in fact supported by the data?  In other words, is
  anything in the paper surprising?''  This is a reasonable comment that is
  extremely difficult to address directly.

  \begin{itemize}
   \item
     We have tried to address this more directly in the paper
     (especially in the introduction and the conclusion); as one example,
     these data could help a tool writer apply the 90-10 rule more easily,
     addressing the relatively small number of preprocessor constructs that
     are used most of the time.  We cannot anticipate all the uses to which
     our data will be put, however.  We do not believe there is one set of
     definitive conclusions to be drawn from the data, but they can be used in
     different ways to support different tasks.
   \item
     The plethora of C code in the world now comprises a natural
     phenomenon; this initial study shows how a key piece of that phenomenon,
     the use of the preprocessor, holds in practice.
   \item
     Even if the data are consistent with what we might have expected,
     confirming this is a useful result (and, as Reviewer 1 indicates, only a
     single data point).
   \item
     The data turn out not to be consistent with readers' assumptions
     about C preprocessor use in real programs; we have provided concrete
     results to explode such assumptions.  What was obvious or intuitive to
     one reader was surprising to another, even among seasoned programmers and
     researchers.  Somewhat to our surprise, each reader has found a different
     aspect of our data and analyses most valuable.  There isn't a single
     element of our analysis that at least one reader hasn't asked for more
     detail about.  (Of course, we cannot hope to extend the paper in every
     direction suggested, but we do welcome followup work.)
   \item
     Many other people who have read the read the paper or heard
     the results have found the results to be of interest and, in some
     respects, surprising.  (This is an admittedly weak argument, but it is
     indeed accurate.)
  \end{itemize}

\item
  For similar reasons to the above, it is difficult to give a succinct
  summary of this mass of data or to describe, in a few words, the
  ``essential results'' that will be of interest to all readers.  We do
  provide some summary information at the beginnings of sections 4, 5, and
  6, but we do not wish to claim that those one-paragraph summaries capture
  everything that is interesting about our analysis.
\end{enumerate}


As a final note, we'd like to thank the reviewers again for their
thoughtful and extensive comments.  And to thank them in advance for
reviewing the extended paper once again.

\closing{Regards,}

\end{letter}

\end{document}
