\begin{description}
\item[Free variables]
  The macro body uses a symbol as a subexpression which is not a formal
  argument, function, macro, typedef, reserved word, or variable defined in
  the macro body.

\item[Assignment]
  The macro body side-effects state via assignment (of the form =, op=, --,
    or ++).

\item[Use macro as type]
  In this macro's body, the result of another macro invocation is used as a
  type.  (This doesn't include all invocations of macros which are
  identified as types, only those uses which are certainly in a context
  where a type is required.)

\item[Pass type as argument~(0.24\%)~]
  In this macro's body, a literal type (such as "int *") is passed to
  another macro.

\item[Pasting]
  The body uses symbol pasting ({\tt \#\#}), which ...

\item[Use argument as type]
  This macro uses one of its arguments as a type.  (Not all uses can be
  identified lexically -- the macro 
\begin{verbatim}
        #define MAKE_DECL(type, name) type name;
\end{verbatim}
  does not necessarily use its first argument as a type, for it might be
  invoked as {\tt \verb|MAKE_DECL|(printf, ("hello world\verb|\|n"))}  or as
  {\tt \verb|MAKE_DECL|(x =, y+z)}.

\item[Self-referential]
  The body refers to its own name.  This feature is used when building a
    wrapper around an existing function [and for other uses?]

\item[Stringization]
  The body uses argument stringization ({\tt \#}), which ...

\item[These next two are for cannibalization]

\item[Stringization and pasting]  The macro body contains {\tt \#} or
  {\tt \#\#}, which treat the macro argument not as a token but as a
  string.  Examples include {\tt \#define spam1(OP,DOC) \verb|{|\#OP, OP,
    1, DOC\verb|}|,}, {\tt \#define REG(xx) register long int xx asm
    (\#xx)}, and {\tt \#define \verb|__CONCAT|(x,y) x \#\# y}.  No C or C++
  language mechanism can replace such macros.

This is a partial lie (I think).
\item[Recursive]  The ISO C standard permits macros to be recursively
  defined (the preprocessor performs only one level of expansion), as in
  {\tt \#define LBIT vcat(LBIT)}.  This mechanism permits already-defined
  or to-be-defined macros to be extended or modified.
\end{description}

