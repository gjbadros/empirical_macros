\backcall{\$token\_\-type}{SzToken}{\$id}
Returns a string identifying what kind of token \textit{id} (an integer) represents.
\backcallobsoleted{(\$cchOffset,\$cbuffersback)}{SumCchExpansionOffset}{}
Gives the source code character position offset and the number of expansions
deep that we currently are.
\backcall{\$cbuffersback}{CbuffersBack}{}
Return the number of macro expansions deep that we currently are in
the current expansion.  This returns 0 if we are not expanding a macro.
The number returned is the number of non-file buffers in the current
stack of expansions (so it is not increased by nested \#include-s).
\backcall{@expanded\_\-macros\_\-list}{MacroExpansionHistory}{}
Returns a list of strings of the form 
\textit{macro\_\-name}\#\textit{arg\_\-num}[\textit{cchOffsetEnd}]. 
Each string explains the contents of a current expansion buffer,
and the first string is the top of the expansion stack.
For example, if there is a single item in the returned list
which is the string "MAX\#1[3]", this means that the current
token came from the expansion of macro MAX and was the first argument
to that macro, ending at character offset 3.  Multiple items
in the returned list mean that a macro expanded to some other
macro which was subsequently expanded.  Argument number 0 signifies
that the token came from the literal body of the macro.
\backcallobsoleted{\$index\_\-of\_\-argument}{ArgOf}{}
Returns the number of the argument from which the current token came.
Prefer using the \$argno parameter passed to the TOKEN hook.
\backcall{\$cch\_\-source\_\-offset}{CchOffset}{}
Returns the character position offset into the current source file.
Use Fname() to get the source file name.
\backcall{\$filename}{InFname}{}
Returns a string giving the name of the main input file (the one
that appeared on the command line.
\backcall{\$filename}{Fname}{}
Returns a string giving the name of the current input file.  This
tracks include-d files and, combined with CchOffset, gives a
unique location in the package.  "@NONE@" returned if no current file.
\backcall{\$filename}{FnameNominal}{}
Returns a string giving the filename corresponding to the current
buffer.  This will be "@NOFILE@" if there is no such file (as is
the case when reading tokens from a macro expansion).
Otherwise it will correspond to Fname().
\backcall{\$encoded\_\-expansion}{ExpansionLookup}{\$macro\_\-name}
Returns the internally-encoded (cpplib-specific) expansion of
the macro with name \texttt{\$macro\_\-name}.  Note that this the string
returned is delimited on either end by "@ " and omits argument occurrences.
"@NOTFOUND@" is return if there is no current macro of the given name.
\backcall{\$cchOutput}{CchOutput}{}
Returns the number of characters already output.  Only non-zero when 
--noparse option is given.
\backcallobsoleted{\$cchCppRead}{CchCppRead}{}
Returns the number of characters that have been read from the
input files.  Prefer using the offsets passed in the various hooks.
\backcall{\$fExpandingMacros}{FExpandingMacros}{}
Returns TRUE iff macros are currently being expanded.
Returns FALSE otherwise.  As arguments are read, macros are not expanded,
and this will return FALSE.
\backcall{@parse\_\-state\_\-list}{ParseStateStack}{}
Returns the current stack of parse state numbers.  The first element is
the top state on the stack.  See the gram.output
file for a listing of what the numbers correspond to.
\backcall{}{SetParseStateStack}{@parse\_\-state\_\-list}
Resets the current stack of parse states so that it is @parse\_\-state\_\-list.
This has the potential to break the parse, and often will outside
of the trivial case of setting the stack of states to what it already is.
\backcall{}{PushBuffer}{\$source\_\-code\_\-string}
Adds \$source\_\-code\_\-string to the being-processed text 
as if it existed in the input at the
current location in the current file.
\backcall{}{EnterScope}{}
Enters a new scoping level; useful in combination
with PushBuffer() for causing code to be parsed without
affecting the current symbol table directly.
\backcall{}{ExitScope}{}
Exits the current scoping level; useful in combination
with PushBuffer() for causing code to be parsed without
affecting the current symbol table directly.
\backcall{}{SetParseDebugging}{}
Set the yydebug flag to TRUE for the underlying parser.
This results in extra state change information being sent to stderr.
It has no affect if --noparse is used.  Initially, parser debugging is off.
\backcall{}{ResetParseDebugging}{}
Reset the yydebug flag to FALSE for the underlying parser.  
This terminates sending extra state change information to stderr.
It has no affect if --noparse is used.
\backcall{}{YYPushStackState}{}
Save the entire current state of the parse stack onto a meta stack.
\backcall{}{YYPopAndRestoreStackState}{}
Restore the entire current state of the parse stack from the meta stack.
\backcall{}{YYPopAndRestoreStackState}{}
Throws out the top state stack from the meta stack.
\backcall{}{YYSwapStackState}{}
Switch the current stack state and the one on the top of the meta stack.
\backcall{}{YYPushDupTopStackState}{}
Push another copy of the top element of the meta stack of stack states onto
the meta stack.
\backcall{\$fStacksEqual}{YYFCompareTopStackState}{}
Return TRUE iff the top of the meta stack of state stacks is
identical to the current state stack. Return FALSE otherwise.
\backcall{\$fDefined}{FLookupSymbol}{\$symbol\_\-id}
Return TRUE iff \$symbol\_\-id is found in the current scope.
Return FALSE otherwise.
