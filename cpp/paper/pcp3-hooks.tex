\hook{STARTUP}{} Called immediately after the perl parser is initialized.
\hook{STARTPARSE}{} Called just before the bison parser is started (i.e., before yyparse()).
\hook{EXIT}{\$return\_\-exit\_\-code} Called just before the system exit() call.
\hook{HANDLE\_\-DIRECTIVE}{\$directive\_\-name} Called exactly once for each directive.  \$directive\_\-name is
the directive as it appeared in the source code with the leading \# removed.
\hook{DO\_\-DEFINE}{\$s\_\-start,\$s\_\-end,\$name\_\-args\_\-body} Called exactly once for each \#define. 
Arguments give the source code character offsets of the line and the 
unprocessed rest of the line including the name of the macro, its arguments
and its definition.
\hook{PRE\_\-DO\_\-UNDEF}{\$s\_\-start,\$s\_\-end,\$mname} Called exactly once for each \#undef.
Arguments give the source code character offsets of the line and the
name of the macro being undefined.  This hook is called just before
the macro is undefined, so its definition still exists 
in the preprocessor's state.
\hook{DO\_\-UNDEF}{\$s\_\-start,\$s\_\-end,\$mname} Called exactly once for each \#undef.
Arguments give the source code character offsets of the line and the
name of the macro being undefined.  This hook is called immediately after
the macro is undefined.
\hook{DO\_\-INCLUDE}{\$s\_\-start,\$s\_\-end,\$filename\_\-given, \$filename\_\-resolved, \$flags} 
Called exactly once for each \#include,
\#include\_\-next, or \#import directive.  Source coded character offsets of
the line are given. Also, \$filename\_\-given is the name of the
file to be included as written in the source;  \$filename\_\-resolved is the
fully qualified path name of the file to be read; and  \$flags is a bitmap
with three relevant masks:  \$ANGLE\_\-BRACKETS, for whether the filename appeared
in angle brackets signifying a system include file; \$SKIP\_\-DIRS, for whether this
is an \#include\_\-next directive; and \$IMPORTING, for whether this is an \#import
directive.
\hook{DO\_\-IF}{\$s\_\-start,\$s\_\-end, \$conditional, \$skipped, \$value} 
Called exactly once for each \#if 
(not \#ifdef or \#ifndef) directive seen by the preprocessor. In particular,
note that normally, nested \#ifXX-s that are ignored will not invoke this hook.
Arguments give the source code character
offsets of the directive; \$conditional is the guard checked, \$skipped is
the literal text that is skipped (conditional was false if this is non-empty), and
\$value is 1 iff the \$conditional evaluated to true (defined), 0 otherwise.
\hook{DO\_\-XIFDEF}{\$s\_\-start,\$s\_\-end, \$kind, \$conditional, \$trailing\_\-garbage, \$skipped, \$fSkipping, \$s\_\-branch\_\-start}
Called exactly once for each \#ifdef or \#ifndef.
The arguments are the same as for the DO\_\-IFDEF hook, except \$kind is either
"IFDEF" or "IFNDEF" depending on the type of conditional; \$fSkipping
replaces the \$value argument and is non-zero iff the conditional fails
(i.e., an IFDEF with a not-defined macro, or an IFNDEF with a defined macro);
\$s\_\-branch\_\-start gives the source code character offset of the start of
the branch taken.
\hook{DO\_\-IFDEF}{\$s\_\-start, \$s\_\-end, \$conditional, \$trailer, \$skipped, \$value}
Called exactly once for each \#ifdef
(not \#if or \#ifndef) directive seen by the preprocessor. In particular,
note that normally, nested \#ifXX-s that are ignored will not invoke this hook.
Arguments give the source code character
offsets of the directive; \$conditional is the name checked for defined-ness;
\$trailer is whatever follows that name on the same line (usually empty); \$skipped is
the literal text that is skipped (conditional was false if this is non-empty), and
\$value is 1 iff the name from the \$conditional was defined, 0 otherwise.
\hook{DO\_\-IFNDEF}{\$s\_\-start, \$s\_\-end, \$conditional, \$trailer, \$skipped, \$value}
Called exactly once for each \#ifndef
(not \#if or \#ifdef) directive seen by the preprocessor. In particular,
note that normally, nested \#ifXX-s that are ignored will not invoke this hook.
Arguments give the source code character
offsets of the directive; \$conditional is the name checked for defined-ness;
\$trailer is whatever follows that name on the same line (usually empty); \$skipped is
the literal text that is skipped (conditional was false if this is non-empty), and
\$value is 1 iff the name from the \$conditional was not defined, 0 otherwise.
\hook{DO\_\-ELSE}{\$s\_\-start, \$s\_\-end, \$orig\_\-conditional, \$trailer, \$skipped, \$fSkipping, \$s\_\-start\_\-branch}
Called exactly once for each \#else directive seen by the preprocessor. In particular,
note that if the directive is skipped due to another outer conditional, this
hook does not get called.
Arguments give the source code character
offsets of the directive; \$orig\_\-conditional is the guard in the
matching \#ifXX directive; 
\$trailer is whatever follows \#else on the same line (usually empty); \$skipped is
the literal text that is skipped (if any); \$fSkipping is 1 iff
this \#else clause's guarded text got skipped; \$s\_\-start\_\-branch is a souce code
character offset of where the branch begins, if it's used, or the character
following the skipped text if not.
\hook{DO\_\-ELIF}{\$s\_\-start, \$s\_\-end, \$already\_\-did\_\-clause, \$conditional, \$skipped, \$value}
Called exactly once for each \#elif directive seen by the preprocessor. In particular,
note that if the directive is skipped due to another outer conditional, this
hook does not get called.
Arguments give the source code character
offsets of the directive; \$already\_\-did\_\-clause is 1 iff one of the
earlier \#ifXX or \#elif-s succeeded (thus implying the code guarded will be skipped);
\$conditional is the conditional tested; \$skipped is the skipped code (if any);
and \$value is what \$conditional evaluated to.  The code guarded is
included iff !\$conditional and \$value, in which case \$skipped will be empty.
\hook{DO\_\-ENDIF}{\$s\_\-start, \$s\_\-end, \$orig\_\-conditional, \$trailer}
Called exactly once for each \#endif directive seen by the preprocessor. In particular,
note that if the directive is skipped due to another outer conditional, this
hook does not get called.
Arguments give the source code character
offsets of the directive; \$orig\_\-conditional is the guard of the matching
\#ifXX directive; \$trailer is the text following the \#endif (often empty);
\hook{CREATE\_\-PREDEF}{\$mname, \$expansion, \$num\_\-args, \$internal\_\-expansion, \$file, \$line, \$r\_\-argnames, \$flags,
\$internal\_\-expansion\_\-args\_\-uses... }
Called once for each predefined macro CPP installs.
\$mname is the name of the macro, \$expansion is its expansion [as literal text], 
\$num\_\-args is the number of arguments it takes, \$internal\_\-expansion is an
internal representation of the expansion (which is interpreted in the 
context of \$internal\_\-expansion\_\-args\_\-uses, see cpplib.c), \$file and \$line
can safely be ignored, \$r\_\-argnames is a comma separated list of the
argument names in reversed order, and \$flags is a bitmask of two flags
\$flags \& \$PREDEFINED is true if this is a predefined macro (always for this hook)
and \$flags \& \$RESTARGS is true if this macro's last argument swallows 
remaining arguments (special gcc cpp feature).
\hook{CREATE\_\-DEF}{\$s\_\-start, \$s\_\-end, \$mname, \$expansion, \$num\_\-args, \$internal\_\-expansion,
\$file, \$line, \$r\_\-argnames, \$flags, \$internal\_\-expansion\_\-args\_\-uses... }
Called once for each macro definition read by  CPP.
\$s\_\-start and \$s\_\-end are the source code character offsets of the definition,
\$mname is the name of the macro, \$expansion is its expansion [as literal text], 
\$num\_\-args is the number of arguments it takes, \$internal\_\-expansion is an
internal representation of the expansion (which is interpreted in the 
context of \$internal\_\-expansion\_\-args\_\-uses, see cpplib.c), \$file and \$line
give the filename and line number where the definition occurs, 
\$r\_\-argnames is a comma separated list of the
argument names in reversed order, and \$flags is a bitmask of two flags
\$flags \& \$PREDEFINED is true if this is a predefined macro (never for this hook)
and \$flags \& \$RESTARGS is true if this macro's last argument swallows 
remaining arguments (special gcc cpp feature).
\hook{DELETE\_\-DEF}{\$mname, \$fExists}
Called once for each macro name that is undefined for any reason (this is
a low level hook called when you attempt to remove a macro name from the 
table of macros,
also HI\_\-DO\_\-UNDEF for the high level hook).  \$mname is the name of the
macro, and \$fExists is 1 iff that macro name used to be defined.
\hook{SPECIAL\_\-SYMBOL}{\$symbol,\$enum\_\-node\_\-type}
Called once for each special symbol (e.g., \_\-\_\-FILE\_\-\_\-) that gets expanded.
\$symbol is the literal text of the symbol, \$enum\_\-node\_\-type is an index
into @enum\_\-node\_\-type giving the C constant corresponding to that symbol
(usually better to just use \$symbol instead).
\hook{EXPAND\_\-MACRO}{\$s\_\-start,\$s\_\-end,\$mname,\$expansion,\$length,\$raw\_\-call,\$has\_\-escapes,\$cbuffersDeep,
\$cnested,@nests,\$cargs,@args}
Called once for each macro expansion in C source (i.e., not for expansions
in \#ifXX guards). \$s\_\-start and \$s\_\-end are the source code character offsets
of the macro invocation; \$mname is name of the macro, \$expansion is what
replaces \$mname in the text (subject to further expansion, of course); \$length
is the number of characters in that expansion, \$raw\_\-call is how the call
appeared in the source (includes the macro name), \$has\_\-escapes is 0 if the
source buffer is actual source text (i.e., if this is a top level expansion),
otherwise it is a 1 (escapes has marked with @- preceding the macro name)
FIXGJB: is that right?; \$cbuffersDeep is how many levels of expansion the text
has undergone beyond the source text (0 means the macro occurrence appeared
directly in the source code); \$cnested tells how many elements are in @nests
(so you can separate its arguments from @args); @nests is obsoleted FIXGJB:;
\$cargs tells how many arguments \$mname takes; and @args contains argument
expansion and use information.  Each argument contributes 7 elements to @args
plus a pair of elements for each use;  e.g., if the first argument was used 3 times
it would have 7 + 3*2 = 13 elements in @args describing its expansion.  The
first 6 arguments are \$raw,\$r\_\-offset,\$expanded,\$e\_\-offset,\$stringified,\$s\_\-offset
which correspond to the raw, expanded, and stringified representations of the
argument.  The \$X\_\-offset elements are internal offsets into a token buffer
and can usually be ignored.  \$stringified may be empty if the argument was
never stringified during the macro expansion.  The seventh element for each
argument tells how many times that argument was used in the expansion, and
successive pairs of elements give character offsets to the appearance of those
arguments in the expansion.  For example if argument 1 is used 3 times, there could
be (3, 5,6, 9,11,  14,18) as the final elements of the @args list.  Note that
because the third expansion is longer than the first two (4 characters instead of
just two), it must have been stringified. FIXGJB is this true?
\hook{MACARG\_\-EXP}{\$mname,\$raw,\$number}
\hook{MACRO\_\-CLEANUP}{\$s\_\-start, \$s\_\-end, \$mname, \$c\_\-nested, @nests}
Called once for each macro expansion buffer as it is removed 
from the stack of text buffers.  \$s\_\-start and \$s\_\-end give the source
code character position offset of the text which expanded into the text
that is in the buffer currently being cleaned and pop-ped;  \$mname is the
name of the macro that resulted in that expansion; \$c\_\-nested and @nests
are obsoleted.  Note that to get the full text of the expansion, you must
track successive strings passed to the CPP\_\-OUT hook, and reset the
string used to accumulate the expansion when the CbuffersBack backcall
is 1 in this hook (meaning you are cleaning up a top level macro expansion).
See cpphook.pm for an example.
\hook{IFDEF\_\-MACRO}{\$s\_\-start,\$s\_\-end,\$mname,\$expansion,\$length,\$raw\_\-call,\$has\_\-escapes,\$cbuffersDeep,
\$cnested,@nests,\$cargs,@args}
Called once for each macro expansion in \#if directives.  The arguments
passed to the hook are the same as those for the EXPAND\_\-MACRO hook, see its
documentation for details.  Note, though, that macros are not usually
expanded in \#ifdef and \#ifndef directives, they are simply tested for
defined-ness by lookup in the symbol table, thus triggering the IFDEF\_\-LOOKUP\_\-MACRO
hook, not this one.  A form such as "\#if FOO" will trigger this hook, however.
\hook{IFDEF\_\-LOOKUP\_\-MACRO}{\$mname,\$f\_\-defined}
Called once for each macro name lookup done during conditional compilation
guard checking.  The arguments are:
\$mname is the macro name being looked up; \$f\_\-defined is non-zero
iff the macro has been defined.  Use backcalls to get the expansion if
it is desired.
\hook{COMMENT}{\$s\_\-start, \$s\_\-end, \$comment\_\-text, \$how\_\-terminated, \$c\_\-lines}
Called once for each comment in the source code seen by preprocessor (i.e.,
comments conditionally-compiiled out will not normally invoke this hook).
\$comment\_\-text is the text of the comment, \$how\_\-term is "*/" for C-style
comments, "nl" for a C++ style comment, or "eof" if the comment
was ended by the end of file.
\$c\_\-lines is the number of lines that the comment spans, including partial lines.
\hook{STRING\_\-CONSTANT}{\$s\_\-start, \$s\_\-end, \$string, \$lines}
Called once for each string in the source code or macro/special symbol expansion.
Note that both stringization and symbols like \_\-\_\-FILE\_\-\_\- will invoke this hook
with their string expansions.
Arguments give the source code character offsets of the string, \$string is
the string, and \$lines is the number of lines (or partial lines) that the
string spans.  \$s\_\-start == \$s\_\-end if the string does not actually appear
in the top level source code (e.g, \_\-\_\-LINE\_\-\_\- or gcc's \_\-\_\-FUNCTION\_\-\_\-).
The source code offsets include the string delimiters, the \$string argument
has them stripped, but has not had backslash sequences replaced.
\hook{CPP\_\-ERROR}{\$nominal\_\-fname,\$line\_\-number,\$message,@extra\_\-args}
Called once each time CPP issues an error.  The hook does not interfere
with normal error handling.  Arguments give the current filename and line
number where the error occurred;  \$message is the string (containing printf
percent-escapes) to be printed, and @extra\_\-args are the values to be 
interpolated into the string.
\hook{CPP\_\-WARN}{\$nominal\_\-fname,\$line\_\-number,\$message,@extra\_\-args}
Called once each time CPP issues a warning.  The hook does not interfere
with normal warning handling.  Arguments give the current filename and line
number where the warning occurred;  \$message is the string (containing printf
percent-escapes) to be printed, and @extra\_\-args are the values to be 
interpolated into the string.
\hook{CPP\_\-PEDWARN}{\$nominal\_\-fname,\$line\_\-number,\$message,@extra\_\-args}
Called once each time CPP issues a pedantic warning.  The hook does not interfere
with normal pedantic warning handling.  Arguments give the current filename and line
number where the pedantic warning occurred;  \$message is the string (containing printf
percent-escapes) to be printed, and @extra\_\-args are the values to be 
interpolated into the string.
\hook{CPP\_\-OUT}{\$string}
Called for each sequence of characters cpp is outputting.  This is the
same text as would go to the output file when using just cpp (as in gcc -E).
Full tokens and/or whitespace are emitted per each call to the hook.
Use the CchOutput() and CchOffset() backcalls for output character offset
and source code character offset, respectively.
\hook{ADD\_\-IMPORT}{\$filename,\$was\_\-found} FIXGJB: 2nd arg right?
Called for each filename that is \#import-ed.  Arguments give that
file name and \$was\_\-found, which is negative iff the file was not found.
\hook{INCLUDE\_\-FILE}{\$filename, \$system\_\-include}
Called for each file that is \#include-d (not necessarily for each \#include
directive; this hook won't be called if the file cannot be found).
Arguments are the filename and \$system\_\-include, which is non-zero if the filename
appears to be a system include file (absolute pathname and in a known
system directory);  it is exactly 2 iff the file is a C-language
system header file for which C++ should assume extern "C".
\hook{DONE\_\-INCLUDE\_\-FILE}{\$filename}
Called upon completion of parsing of an included file.  The filename
that what just finished being parsed is the only argument.
\hook{TOKEN}{\$token,\$raw,\$macro\_\-name,\$arg\_\-num,@history}
Called once for each token CPP reads.  \$token is the type of token,
with a leading CPP\_\-; \$raw is the raw characters which constitute that
token; \$macro\_\-name is the name of the macro that expanded to create this token,
or is empty if the token did not result from macro expansion; \$arg\_\-num is -1
if the token did not come from a macro expansion, 0 if it came from the
body of a macro expansion (i.e., not an argument), or positive and equal
to the argument number that produced the token, if from an substitution
of an argument in a macro expansion; @history is obsoleted, use the
MacroExpansionHistory() backcall instead.  Note that the token "CPP\_\-POP"
means that an expansion buffer has been popped from the stack of expansions,
and is not a real lexical token.  This may be removed from future version
so use the POP\_\-BUFFER hook to handle those events.
\hook{FUNCTION}{\$name,\$fStatic}
Called once after an entire function is parsed.  Arguments
give the name of the newly defined function, and \$fStatic is non-zero
iff the function was declared to be static (i.e. not global).
\hook{FUNC\_\-CALL}{\$name}
Called once after each function call is parsed.  Only argument is
the name of the function being called.
\hook{ANNOTATE}{} Obsoleted
\hook{POP\_\-BUFFER}{\$cbuffersDeep}
Called once for each buffer that is popped off of the stack of
buffers to be parsed. The only argument is the new (after the pop)
number of non-file buffers
deep the stack is.  In particular, when that number is 0, 
the parser is not parsing a macro expansion any longer.
