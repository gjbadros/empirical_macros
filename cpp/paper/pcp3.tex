%-------------------------------------------------------------------------
%  paper.tex -- The basic outline of the paper, includes all the sections
%
%   $Id$
%-------------------------------------------------------------------------


\documentclass{article}
\usepackage{fullpage}
%\usepackage{fancybox}
\usepackage{graphicx}
\usepackage{rotating}
\usepackage{lscape}

\newcommand{\pcp}{\mbox{\textsf{PCP}$^3$}}
\newcommand{\pcppp}{\mbox{\textsf{PCppP}}}
\newcommand{\Cpp}{\mbox{\textsf{cpp}}}
\newcommand{\CPP}{\mbox{\textsf{C++}}}
\newcommand{\Perl}{\mbox{\textsf{Perl}}}
\newcommand{\C}{\mbox{\textsf{C}}}

\newcommand{\backcall}[3]{\item \texttt{#2} (\texttt{#3})
  \textbf{returns} \texttt{#1} \\ }

%\newcommand{\backcallobsoleted}[3]{\item \texttt{#2} (\texttt{#3})
%  \textbf{returns} \texttt{#1} \textsc{Obsoleted}\\ }

% Don't even print these out for the paper
\newcommand{\backcallobsoleted}[3]{}

\newcommand{\hook}[2]{\item \texttt{#1} (\texttt{#2}) \\ }

\newcommand{\hookobsoleted}[2]{}

\newcommand{\ppd}[1]{\texttt{\##1}}

\newcommand{\file}[1]{{\small \texttt{#1}}}
\newcommand{\email}[1]{{\small \texttt{#1}}}
\newcommand{\program}[1]{{\small \texttt{#1}}}
\newcommand{\syscall}[1]{{\textsf{#1}}}
\newcommand{\sectionref}[1]{section \ref{#1}, on page \pageref{#1}}
\newcommand{\ie}{i.e.,}
\newcommand{\eg}{e.g.,}
\newcommand{\etc}{etc\.}
\newcommand{\figref}[1]{Figure~\ref{#1}}

\title{\pcp{}: A \C{} Front End for \\ Preprocessor Analysis and Transformation}
\author{Greg J. Badros%
  \thanks{Supported by a National Science Foundation
    Graduate Fellowship. Any opinions, findings, conclusions, or
    recommendations expressed in this publication are those of the
    author, and do not necessarily reflect the views of the National
    Science Foundation.}
  }

\begin{document}
\maketitle

\begin{abstract}
\label{sec:abstract}
  Though the \C{} preprocessor provides necessary language features, it
  does so in a completely unstructured way.  The lexical nature of
  \Cpp{} creates numerous problems for software engineers and their
  tools, all stemming from the chasm between the engineer's view of the
  source code and the compiler's view.  The simplest way to reduce this
  problem is to minimize use of the preprocessor.  In light of the data
  collected in a prior empirical analysis, this study considers some
  simple transformations on preprocessor use in \C{} code to \CPP{}
  language features which could substantially reduce use of \Cpp{} in
  legacy code. Existing tools for analyzing \C{} source including its
  preprocessor directives are unsuitable for such transformations, so
  this work introduces a new approach: tightly integrating the
  preprocessor with a \C{} language parser, permitting the code to be
  analyzed at both the preprocessor and syntactic levels simultaneously.
  The front-end framework, called \pcp{}, couples arbitrary \Perl{}
  ``hooks'' upon various preprocessor and parser events and is thus
  general and flexible. \pcp{}'s strengths and weaknesses are discussed
  in the context of several program understanding and transformation
  tools, including a conservative analysis to support replacing \Cpp{}'s
  \ppd{define} directives with \CPP{} language features.

\end{abstract}
\bigskip

% Introduction, including problem statement, goals, etc
\section{Introduction}
\label{sec:intro}
More than twenty years ago, Dennis Ritchie designed the \C{} language to
include a textual macro preprocessor called
\Cpp{}~\cite[Ch.~3]{Harbison91}.  Given the simplicity of the language
and the state of the art in compiler technology in the mid-1970s, his
decision to provide some language features in this extra-linguistic tool
may have been justified.  For the last couple of decades, \C{} programs
have exploited \Cpp{}'s capabilities for everything from manifest
constants and type-less pseudo-inline functions, through modularization
and symbol generation.  Bjarne Stroustrup, the designer and original
implementor of \CPP{}, notes that ``without the \C{} preprocessor, \C{}
itself . . . would have been stillborn''~\cite[p.~119]{Stroustrup94}.
Most certainly \Cpp{} contributes greatly to \C{}'s expressiveness and
portability, but perhaps at too large a cost.  Stroustrup recognizes
this tradeoff:

\begin{quotation}
Occasionally, even the most extreme uses of \Cpp{} are useful, but its
facilities are so unstructured and intrusive that they are a constant
problem to programmers, maintainers, people porting code, and tool
builders~\cite[p.~424]{Stroustrup94}.
\end{quotation}

%% FIXGJB: is \ldots too cute?
\subsection{Why \Cpp{} is good \ldots and bad}

The intrinsic problem with \Cpp{} is also its fundamental strength: the
conceptually distinct first pass of textual processing over \C{} source
code.  This introduces a giant chasm between the code that the
programmer sees and what the compiler-proper (\ie{} the \C{} compiler
separate from the preprocessor) ultimately compiles.\footnote{To avoid
  ambiguity, we will use \emph{preprocessed} to refer to the view of a
  translation unit after running \Cpp{} on it, and \emph{unprocessed} to
  refer to the original source code.} Consider the program in
\figref{fig:badmain}\footnote{Adapted from an example given by
  Stroustrup~\cite[p.~423]{Stroustrup94}.}  Though a legal ANSI \C{}
program, its semantics are undefined in light of \Cpp{}.  When compiled
using \texttt{cc -Dprintf(x)=}, the program no longer outputs the
desired ``Hello world'' message.

\begin{figure}[hbt]
\begin{center}
\begin{minipage}[t]{4.5in}
\label{fig:badmain}
\begin{verbatim}
#include <stdio.h>

int main(int argc, char *argv[]) {
  printf("Hello world");
  return 0;
}
\end{verbatim}
\caption{\texttt{main.c}: An example highlighting incomplete semantics of
  \C{} source code due to \Cpp{}}
\end{minipage}
\end{center}
\end{figure}

Experienced and novice \C{} programmers alike have been frustrated by
similar misunderstandings of source code due to the arbitrary
transformations \Cpp{} performs before the language compiler ever is
invoked.\footnote{Some more modern \C{} compilers integrate the
  preprocessor with the lexical scanning phase, but, by necessity, it is
  still a separate first phase conceptually.}  Such confusions are
easily eliminated by allowing the software engineer to see the code
exactly as the compiler does.  Unfortunately, that view of the program
is a level of abstraction lower than the un-preprocessed source
provides.  Well understood identifiers such as \texttt{stderr} appear as
the far less readable
\texttt{(\_IO\_FILE*)(\&\_IO\_stderr\_)}\footnote{This is the output
  generated when preprocessed using the \texttt{gcc} compiler's standard
  include files.}, and useful encapsulations such as \texttt{assert}
degenerate into nonsense sequences of symbols.  Worsening matters,
preprocessing code eliminates conditional compilation artifacts that are
essential to the portability and versatility of the source code
(see~\cite{Krone94}), and multiplies the length of translation units by
in-line inclusion of thousands of lines of standard header files.

\subsection{What this means for software tools}

Though software engineers are rarely encouraged to work directly with
preprocessed code, the majority of software engineering tools operate on
exactly that view of the source.  Debuggers, call graph extractors, data
flow analyzers, ASTLog [[FIXGJB: REF?]], and countless other tools
either run \Cpp{} as the first stage in their analysis, or use
representations derived from a compiler operating on the preprocessed
code.  Though the parser or abstract-syntax-tree based approach these
tools apply results in information that is precise for the input they
consider, their usefulness for human-targeted program understanding is
diminished due to the textual transformations from
preprocessing.\footnote{In contrast, this approach is exactly right for
  the compiler which has little need of preserving high level
  abstractions in generating object code.}  Additionally, parsing
requires a syntactically correct program and all header files to
exist---these constraints are not realistic during many software
maintenance activities (\eg{} porting, switching compilers, \etc{}).

Some tools choose a different tradeoff and operate instead on the
un-preprocessed source code exactly as the programmer sees it.  These
tools cannot use a straightforward parser or construct an accurate
abstract syntax tree because the source is not syntactically-correct
\C{} code.  Lexical tools (\eg{} etags, LLSME~\cite{Murphy95}, and
font-lock-mode for emacs) and approximate parsers (\eg{} \texttt{a*} and
\texttt{Genoa}~\cite{Griswold96}, \texttt{LCLint} [[FIXGJB:
True?]]~cite{LCLint}, \etc{}) use this approach.  In general, this leads
to greater speed (especially for the lexical approach) and improved
robustness to syntax errors and language variants (approximate parsers
generally lose at worst the current top level construct, and often can
do better). Additionally, because the input is un-preprocessed, the
extracted information is presented to the human software engineer at the
same level of abstraction as the source with which she is working.
Unfortunately, by disregarding (or only partially honoring) the \Cpp{}
directives, the extracted model of the source code can only be an
approximation of the program's semantics to a full compiler.  Macros, an
especially flexible ``feature'' of \Cpp{}, can wreak havoc on this
approach by hiding function calls in pseudo-inline functions or
customizing syntax of declarations or scoping structure.  Griswold and
Atkinson studied mistakes in call graph extraction using various tools
and found that macro expansion was a major cause of both false positives
and false negatives~\cite{Griswold96}.  Such tools, though, are
inappropriate for software tools that require exact or conservative
information.

An obvious solution to the problems \Cpp{} presents is to just avoid it
(and therefore, \C{}, as we have already seen how strong \C{} relies on
it) entirely.  By using another language which directly provides
language features such as modules, constants, inline functions, generic
functions, and other higher level abstractions that \C{} emulates via
its preprocessor, most of the above issues are avoided.  Casting away
\C{} in favor of most modern languages would necessitate discarding
billions of lines of useful legacy code.  The one notable exception is
\CPP{}~\cite{CD2DraftStandard}. For compatibility with \C{}, \CPP{}
remains encumbered by \Cpp{}, yet \CPP{} does provide language-level
support for many higher level constructs, thus making numerous \Cpp{}
constructs redundant\footnote{In fact, this was an explicit design goal
  for Stroustrup~\cite[p.~424]{Stroustrup94}.}  Thus, migrating \C{}
code to \CPP{} potentially provides a path to reduce usage of the
preprocessor, and thus make tools' and programmers' jobs easier.

\subsection{Outline}

The next section highlights significant findings from a prior empirical
study of \Cpp{} use~\cite{EmpCpp-TR} with an eye toward understanding
what fraction of preprocessor use in existing software artifacts can be
replaced by \CPP{} language features.  Section~\ref{sec:pcp3} describes
the tool the author developed to support accurate analysis of
un-preprocessed code without losing the high level abstractions
utilizing the preprocessor. Section~\ref{sec:results} illustrates some
of the analyses that \pcp{} does well, and discusses its support for
automating safe conversion of some simple \Cpp{} uses into \CPP{} language
features. Section~\ref{sec:related} describes some related work, and
section~\ref{sec:summary} discusses the contribution of this work, some
of its shortcomings, and some areas for possible future work.


%% Is C++ expressive enough?
% Background
\section{Feasibility of Transformation}
\label{sec:feasibility}
Many \Cpp{} constructs clearly have analogues in newer language
features.  For example, \ppd{define}s of simple numeric and string
constants can often be replaced with enumeration declarations or static
constant variable declarations (both newer features of ANSI
\C{}, not just of \CPP{}).
%FIXGJB: interesting that this language construct is used so little
Some function-like \ppd{define}s of pseudo-inline functions can
correspond to \CPP{}'s real \texttt{inline} functions (perhaps made
generic through use of a template).  An optimizing compiler performing
trivial dead code elimination can effect the same result as a
conditional compilation directive guarding debug-only code (e.g.,
replacing a syntactically correct \texttt{\ppd{ifndef} NDEBUG} and
\ppd{endif} pair with a simple \texttt{if (!fDebug) \{} and
\texttt{\}}).

%% Types of questions the analysis must support answering


% PCP^3
\section{The \pcp{} Infrastructure}
\label{sec:pcp3}
\pcp{} is built from three software components: a \texttt{P}arser, a \texttt{C}
\texttt{P}re\texttt{p}rocessor, and a \texttt{P}erl action
language.\footnote{Hence its name, \pcppp{}, shortened to \pcp{}.}

\subsection{Parser}

The first major component of \pcp{} is an ANSI \C{} compatible parser.
Choosing a parser was more difficult than selecting a preprocessor. Not
only are there many more freely available parsers, they also tend to be
more tightly coupled to their back-end, thus complicating reuse.
Ultimately, the parser from \texttt{CTree}, a freely available \C{}
front end which creates an abstract syntax tree from a preprocessed
input file, was embedded in \pcp{}.  This parser has the advantage of
doing little more than parsing and creating an AST.  Its lexer and
parser both are mechanically generated from \texttt{flex}~\cite{Flex}
(freely available implementation of \texttt{lex}) and
\texttt{bison}~\cite{Bison} specifications, respectively.
\texttt{CTree} also implements a simply, but fully scoped symbol table,
relieving \pcp{} of another essential duty.

\subsection{Preprocessor}

As a software tool targeting \C{} code, the design of \pcp{} faced the
same difficulties as other tools as outlined in section~\ref{sec:intro}.
Disregarding the preprocessor is clearly not an option since analyzing
the preprocessor is integral to \pcp{}.  But approximating \Cpp{} is not
good enough either, as it is essential that the tool mimic \Cpp{}
exactly. Thus, the \C{} preprocessing library from the GNU \C{}
compiler's (\texttt{gcc}) well-tested (and slightly extended)
\Cpp{}~\cite{GCC} is embedded in \pcp{}.  By using \texttt{gcc}'s
\texttt{cpplib}, \pcp{} is able to ``see'' both the un-preprocessed and
the preprocessed views of source code.  Though \Cpp{} itself inserts
\ppd{line} directives in its output to provide a rudimentary mapping
between the two views, \pcp{} keeps a much finer mapping, tracking
what tokens came from source code directly, from expansions of macro
bodies, arguments of macros, \etc.\footnote{The tight mapping is similar
  to the approach the intentional programming group at Microsoft
  Research is using trying to recover preprocessor ``intentions'' from
  fully preprocessed code annotated with this mapping
  information~cite{MSIPPersonal}.}  By connecting the annotated tokens
and exposing the state of the preprocessor to the parser and vice-versa,
the communication between \Cpp{} and the \C{} language parser is widened
from the thin straw of tokens passed via a Unix pipe that plagues other
tools choosing to operate on preprocessed code.

\subsection{Perl action language}

Griswold and Atkinson note that using a special-purpose interpreted
action language helped the scalability of various software tools in
extracting a call graph~\cite{Griswold96}.  Interpreted languages can
speed development time, especially when prototyping analyses.  For
\pcp{}, \Perl{}~\cite{Perl} was chosen because it interfaces
easily with \C{}, it is in widespread use, and the author is comfortable
with it.\footnote{Additionally, the tools used in the empirical study of
  \Cpp{}~\cite{EmpCpp-TR} use are written in \Perl{}, thus
  presenting yet another opportunity for reuse.}

The \Perl{} interface of \pcp{} is split into two parts:

%%FIXGJB: did the numbers of hooks/backcalls change?
\begin{itemize}
\item Action ``hooks'' written in \Perl{} that \C{} code in \pcp{}
      invoke on various occurrences. Each hook is directly passed (via
      normal \Perl{} passing conventions) a set of parameters relevant
      to the current action.  Example actions include the scanning of
      preprocessor directives, the creation of a macro definition, the
      expansion (\ie{} use) of a macro name, and the parsing of a
      variable declaration.  There are forty-one hooks presently
      specified in \pcp{}. See \sectionref{sec:hooks} for details.
\item Subroutine ``backcalls'' written in \C{} (actually,
      \texttt{PerlXS} the dialect of \C{} used for \Perl{} extensions)
      that the \Perl{} hooks are free to call to access pre-specified
      \C{} data structures or actively interact with the Parser or
      \Cpp{} components of \pcp{}.  Example subroutines include getting
      the name of the currently-processed file, inserting arbitrary code
      for the preprocessor to parse, and instructing the parser to enter
      a new scope.  There are twenty-six backcalls presently permitted
      by \pcp{}.  See \sectionref{sec:backcalls} for details.
\end{itemize}

\noindent The \Perl{} code is free to manage its local data structures
      arbitrarily, and may import modules as an ordinary \Perl{} program
      would.  The \C{} data structures are protected behind the hooks
      and backcalls interfaces.

\subsection{A simple example}



% Benchmarks, results, etc.
\section{Results}
\label{sec:results}

% Summary & Future work
\section{Related Work}
\label{sec:related}

% Summary & Future work
\section{Summary and Future Work}
\label{sec:summary}

\appendix
\newpage
\section{Action hooks called by \pcp}
\begin{footnotesize}
\label{sec:hooks}
\begin{itemize}
\sloppy
\hook{STARTUP}{} Called immediately after the perl parser is initialized.
\hook{STARTPARSE}{} Called just before the bison parser is started (i.e., before yyparse()).
\hook{EXIT}{\$return\_\-exit\_\-code} Called just before the system exit() call.
\hook{HANDLE\_\-DIRECTIVE}{\$directive\_\-name} Called exactly once for each directive.  \$directive\_\-name is
the directive as it appeared in the source code with the leading \pphash{} removed.
\hook{DO\_\-DEFINE}{\$s\_\-start, \$s\_\-end, \$name\_\-args\_\-body} Called exactly once for each \pphash{}define. 
Arguments give the source code character offsets of the line and the 
unprocessed rest of the line including the name of the macro, its arguments
and its definition.
\hook{PRE\_\-DO\_\-UNDEF}{\$s\_\-start, \$s\_\-end, \$mname} Called exactly once for each \pphash{}undef.
Arguments give the source code character offsets of the line and the
name of the macro being undefined.  This hook is called just before
the macro is undefined, so its definition still exists 
in the preprocessor's state.
\hook{DO\_\-UNDEF}{\$s\_\-start, \$s\_\-end, \$mname} Called exactly once for each \pphash{}undef.
Arguments give the source code character offsets of the line and the
name of the macro being undefined.  This hook is called immediately after
the macro is undefined.
\hook{DO\_\-INCLUDE}{\$s\_\-start, \$s\_\-end, \$filename\_\-given, \$filename\_\-resolved, \$flags} 
Called exactly once for each \pphash{}include,
\pphash{}include\_\-next, or \pphash{}import directive.  Source coded character offsets of
the line are given. Also, \$filename\_\-given is the name of the
file to be included as written in the source;  \$filename\_\-resolved is the
fully qualified path name of the file to be read; and  \$flags is a bitmap
with three relevant masks:  \$ANGLE\_\-BRACKETS, for whether the filename appeared
in angle brackets signifying a system include file; \$SKIP\_\-DIRS, for whether this
is an \pphash{}include\_\-next directive; and \$IMPORTING, for whether this is an \pphash{}import
directive.
\hook{DO\_\-IF}{\$s\_\-start, \$s\_\-end, \$conditional, \$skipped, \$value} 
Called exactly once for each \pphash{}if 
(not \pphash{}ifdef or \pphash{}ifndef) directive seen by the preprocessor. In particular,
note that normally, nested \pphash{}ifXX-s that are ignored will not invoke this hook.
Arguments give the source code character
offsets of the directive; \$conditional is the guard checked, \$skipped is
the literal text that is skipped (conditional was false if this is non-empty), and
\$value is 1 iff the \$conditional evaluated to true (defined), 0 otherwise.
\hook{DO\_\-XIFDEF}{\$s\_\-start, \$s\_\-end, \$kind, \$conditional, \$trailing\_\-garbage, \$skipped, \$fSkipping, \$s\_\-branch\_\-start}
Called exactly once for each \pphash{}ifdef or \pphash{}ifndef.
The arguments are the same as for the DO\_\-IFDEF hook, except \$kind is either
"IFDEF" or "IFNDEF" depending on the type of conditional; \$fSkipping
replaces the \$value argument and is non-zero iff the conditional fails
(i.e., an IFDEF with a not-defined macro, or an IFNDEF with a defined macro);
\$s\_\-branch\_\-start gives the source code character offset of the start of
the branch taken.  Note that this hook is called after the DO\_\-IFDEF or DO\_\-IFNDEF
hook.
\hook{DO\_\-IFDEF}{\$s\_\-start, \$s\_\-end, \$conditional, \$trailer, \$skipped, \$value}
Called exactly once for each \pphash{}ifdef
(not \pphash{}if or \pphash{}ifndef) directive seen by the preprocessor. In particular,
note that normally, nested \pphash{}ifXX-s that are ignored will not invoke this hook.
Arguments give the source code character
offsets of the directive; \$conditional is the name checked for defined-ness;
\$trailer is whatever follows that name on the same line (usually empty); \$skipped is
the literal text that is skipped (conditional was false if this is non-empty), and
\$value is 1 iff the name from the \$conditional was defined, 0 otherwise.
\hook{DO\_\-IFNDEF}{\$s\_\-start, \$s\_\-end, \$conditional, \$trailer, \$skipped, \$value}
Called exactly once for each \pphash{}ifndef
(not \pphash{}if or \pphash{}ifdef) directive seen by the preprocessor. In particular,
note that normally, nested \pphash{}ifXX-s that are ignored will not invoke this hook.
Arguments give the source code character
offsets of the directive; \$conditional is the name checked for defined-ness;
\$trailer is whatever follows that name on the same line (usually empty); \$skipped is
the literal text that is skipped (conditional was false if this is non-empty), and
\$value is 1 iff the name from the \$conditional was not defined, 0 otherwise.
\hook{DO\_\-ELSE}{\$s\_\-start, \$s\_\-end, \$orig\_\-conditional, \$trailer, \$skipped, \$fSkipping, \$s\_\-start\_\-branch}
Called exactly once for each \pphash{}else directive seen by the preprocessor. In particular,
note that if the directive is skipped due to another outer conditional, this
hook does not get called.
Arguments give the source code character
offsets of the directive; \$orig\_\-conditional is the guard in the
matching \pphash{}ifXX directive; 
\$trailer is whatever follows \pphash{}else on the same line (usually empty); \$skipped is
the literal text that is skipped (if any); \$fSkipping is 1 iff
this \pphash{}else clause's guarded text got skipped; \$s\_\-start\_\-branch is a souce code
character offset of where the branch begins, if it's used, or the character
following the skipped text if not.
\hook{DO\_\-ELIF}{\$s\_\-start, \$s\_\-end, \$already\_\-did\_\-clause, \$conditional, \$skipped, \$value}
Called exactly once for each \pphash{}elif directive seen by the preprocessor. In particular,
note that if the directive is skipped due to another outer conditional, this
hook does not get called.
Arguments give the source code character
offsets of the directive; \$already\_\-did\_\-clause is 1 iff one of the
earlier \pphash{}ifXX or \pphash{}elif-s succeeded (thus implying the code guarded will be skipped);
\$conditional is the conditional tested; \$skipped is the skipped code (if any);
and \$value is what \$conditional evaluated to.  The code guarded is
included iff !\$conditional and \$value, in which case \$skipped will be empty.
\hook{DO\_\-ENDIF}{\$s\_\-start, \$s\_\-end, \$orig\_\-conditional, \$trailer}
Called exactly once for each \pphash{}endif directive seen by the preprocessor. In particular,
note that if the directive is skipped due to another outer conditional, this
hook does not get called.
Arguments give the source code character
offsets of the directive; \$orig\_\-conditional is the guard of the matching
\pphash{}ifXX directive; \$trailer is the text following the \pphash{}endif (often empty);
\hook{CREATE\_\-PREDEF}{\$mname, \$expansion, \$num\_\-args, \$internal\_\-expansion, \$file, \$line, \$r\_\-argnames, \$flags, \$internal\_\-expansion\_\-args\_\-uses... }
Called once for each predefined macro CPP installs.
\$mname is the name of the macro, \$expansion is its expansion [as literal text], 
\$num\_\-args is the number of arguments it takes, \$internal\_\-expansion is an
internal representation of the expansion (which is interpreted in the 
context of \$internal\_\-expansion\_\-args\_\-uses, see cpplib.c), \$file and \$line
can safely be ignored, \$r\_\-argnames is a comma separated list of the
argument names in reversed order, and \$flags is a bitmask of two flags
\$flags \& \$PREDEFINED is true if this is a predefined macro (always for this hook)
and \$flags \& \$RESTARGS is true if this macro's last argument swallows 
remaining arguments (special gcc cpp feature).
\hook{CREATE\_\-DEF}{\$s\_\-start, \$s\_\-end, \$mname, \$expansion, \$num\_\-args, \$internal\_\-expansion, \$file, \$line, \$r\_\-argnames, \$flags, \$internal\_\-expansion\_\-args\_\-uses... }
Called once for each macro definition read by  CPP.
\$s\_\-start and \$s\_\-end are the source code character offsets of the definition,
\$mname is the name of the macro, \$expansion is its expansion [as literal text], 
\$num\_\-args is the number of arguments it takes, \$internal\_\-expansion is an
internal representation of the expansion (which is interpreted in the 
context of \$internal\_\-expansion\_\-args\_\-uses, see cpplib.c), \$file and \$line
give the filename and line number where the definition occurs, 
\$r\_\-argnames is a comma separated list of the
argument names in reversed order, and \$flags is a bitmask of two flags
\$flags \& \$PREDEFINED is true if this is a predefined macro (never for this hook)
and \$flags \& \$RESTARGS is true if this macro's last argument swallows 
remaining arguments (special gcc cpp feature).
\hook{DELETE\_\-DEF}{\$mname, \$fExists}
Called once for each macro name that is undefined for any reason (this is
a low level hook called when you attempt to remove a macro name from the 
table of macros,
also HI\_\-DO\_\-UNDEF for the high level hook).  \$mname is the name of the
macro, and \$fExists is 1 iff that macro name used to be defined.
\hook{SPECIAL\_\-SYMBOL}{\$symbol, \$enum\_\-node\_\-type}
Called once for each special symbol (e.g., \_\-\_\-FILE\_\-\_\-) that gets expanded.
\$symbol is the literal text of the symbol, \$enum\_\-node\_\-type is an index
into @enum\_\-node\_\-type giving the C constant corresponding to that symbol
(usually better to just use \$symbol instead).
\hook{EXPAND\_\-MACRO}{\$s\_\-start, \$s\_\-end, \$mname, \$expansion, \$length, \$raw\_\-call, \$has\_\-escapes, \$cbuffersDeep, \$cnested, @nests, \$cargs, @args}
Called once for each macro expansion in C source (i.e., not for expansions
in \pphash{}ifXX guards). \$s\_\-start and \$s\_\-end are the source code character offsets
of the macro invocation; \$mname is name of the macro, \$expansion is what
replaces \$mname in the text (subject to further expansion, of course); \$length
is the number of characters in that expansion, \$raw\_\-call is how the call
appeared in the source (includes the macro name), \$has\_\-escapes is 0 if the
source buffer is actual source text (i.e., if this is a top level expansion),
otherwise it is a 1 (escapes has marked with @- preceding the macro name);
\$cbuffersDeep is how many levels of expansion the text
has undergone beyond the source text (0 means the macro occurrence appeared
directly in the source code); \$cnested tells how many elements are in @nests
(so you can separate its arguments from @args); @nests is non-empty when this
expansion is an argument to another macro---see the body of the paper for 
a complete explanation.
\$cargs tells how many arguments \$mname takes; and @args contains argument
expansion and use information.  Each argument contributes 7 elements to @args
plus a pair of elements for each use;  e.g., if the first argument was used 3 times
it would have 7 + 3*2 = 13 elements in @args describing its expansion.  The
first 6 arguments are \$raw,\$r\_\-offset,\$expanded,\$e\_\-offset,\$stringified,\$s\_\-offset
which correspond to the raw, expanded, and stringified representations of the
argument.  The \$X\_\-offset elements are internal offsets into a token buffer
and can usually be ignored.  \$stringified may be empty if the argument was
never stringified during the macro expansion.  The seventh element for each
argument tells how many times that argument was used in the expansion, and
successive pairs of elements give character offsets to the appearance of those
arguments in the expansion.  For example if argument 1 is used 3 times, there could
be (3, 5,6, 9,11,  14,18) as the final elements of the @args list.  Note that
because the third expansion is longer than the first two (4 characters instead of
just two), it must have been stringified.
\hookobsoleted{MACARG\_\-EXP}{\$mname, \$raw, \$number}
\hook{MACRO\_\-CLEANUP}{\$s\_\-start, \$s\_\-end, \$mname, \$c\_\-nested, @nests}
Called once for each macro expansion buffer as it is removed 
from the stack of text buffers.  \$s\_\-start and \$s\_\-end give the source
code character position offset of the text which expanded into the text
that is in the buffer currently being cleaned and pop-ped;  \$mname is the
name of the macro that resulted in that expansion; \$c\_\-nested and @nests
are obsoleted.  Note that to get the full text of the expansion, you must
track successive strings passed to the CPP\_\-OUT hook, and reset the
string used to accumulate the expansion when the CbuffersBack backcall
is 1 in this hook (meaning you are cleaning up a top level macro expansion).
See cpphook.pm for an example.
\hook{IFDEF\_\-MACRO}{\$s\_\-start, \$s\_\-end, \$mname, \$expansion, \$length, \$raw\_\-call, \$has\_\-escapes, \$cbuffersDeep, \$cnested, @nests, \$cargs, @args}
Called once for each macro expansion in \pphash{}if directives.  The arguments
passed to the hook are the same as those for the EXPAND\_\-MACRO hook, see its
documentation for details.  Note, though, that macros are not usually
expanded in \pphash{}ifdef and \pphash{}ifndef directives, they are simply tested for
defined-ness by lookup in the symbol table, thus triggering the IFDEF\_\-LOOKUP\_\-MACRO
hook, not this one.  A form such as "\pphash{}if FOO" will trigger this hook, however.
\hook{IFDEF\_\-LOOKUP\_\-MACRO}{\$mname, \$f\_\-defined}
Called once for each macro name lookup done during conditional compilation
guard checking.  The arguments are:
\$mname is the macro name being looked up; \$f\_\-defined is non-zero
iff the macro has been defined.  Use backcalls to get the expansion if
it is desired.
\hook{COMMENT}{\$s\_\-start, \$s\_\-end, \$comment\_\-text, \$how\_\-terminated, \$c\_\-lines}
Called once for each comment in the source code seen by preprocessor (i.e.,
comments conditionally-compiiled out will not normally invoke this hook).
\$comment\_\-text is the text of the comment, \$how\_\-term is "*/" for C-style
comments, "nl" for a C++ style comment, or "eof" if the comment
was ended by the end of file.
\$c\_\-lines is the number of lines that the comment spans, including partial lines.
\hook{STRING\_\-CONSTANT}{\$s\_\-start, \$s\_\-end, \$string, \$lines}
Called once for each string in the source code or macro/special symbol expansion.
Note that both stringization and symbols like \_\-\_\-FILE\_\-\_\- will invoke this hook
with their string expansions.
Arguments give the source code character offsets of the string, \$string is
the string, and \$lines is the number of lines (or partial lines) that the
string spans.  \$s\_\-start == \$s\_\-end if the string does not actually appear
in the top level source code (e.g, \_\-\_\-LINE\_\-\_\- or gcc's \_\-\_\-FUNCTION\_\-\_\-).
The source code offsets include the string delimiters, the \$string argument
has them stripped, but has not had backslash sequences replaced.
\hook{CPP\_\-ERROR}{\$nominal\_\-fname, \$line\_\-number, \$message, @extra\_\-args}
Called once each time CPP issues an error.  The hook does not interfere
with normal error handling.  Arguments give the current filename and line
number where the error occurred;  \$message is the string (containing printf
percent-escapes) to be printed, and @extra\_\-args are the values to be 
interpolated into the string.
\hook{CPP\_\-WARN}{\$nominal\_\-fname, \$line\_\-number, \$message, @extra\_\-args}
Called once each time CPP issues a warning.  The hook does not interfere
with normal warning handling.  Arguments give the current filename and line
number where the warning occurred;  \$message is the string (containing printf
percent-escapes) to be printed, and @extra\_\-args are the values to be 
interpolated into the string.
\hook{CPP\_\-PEDWARN}{\$nominal\_\-fname, \$line\_\-number, \$message, @extra\_\-args}
Called once each time CPP issues a pedantic warning.  The hook does not interfere
with normal pedantic warning handling.  Arguments give the current filename and line
number where the pedantic warning occurred;  \$message is the string (containing printf
percent-escapes) to be printed, and @extra\_\-args are the values to be 
interpolated into the string.
\hook{CPP\_\-OUT}{\$string}
Called for each sequence of characters cpp is outputting.  This is the
same text as would go to the output file when using just cpp (as in gcc -E).
Full tokens and/or whitespace are emitted per each call to the hook.
Use the CchOutput() and CchOffset() backcalls for output character offset
and source code character offset, respectively.
\hook{ADD\_\-IMPORT}{\$filename, \$filedes\_\-num}
Called for each filename that is \pphash{}import-ed.  Arguments give that
file name and the number of the file-descriptor that file was opened
on (negative iff the file was not found).
\hook{INCLUDE\_\-FILE}{\$filename, \$system\_\-include}
Called for each file that is \pphash{}include-d (not necessarily for each \pphash{}include
directive; this hook won't be called if the file cannot be found).
Arguments are the filename and \$system\_\-include, which is non-zero if the filename
appears to be a system include file (absolute pathname and in a known
system directory);  it is exactly 2 iff the file is a C-language
system header file for which C++ should assume extern "C".
\hook{DONE\_\-INCLUDE\_\-FILE}{\$filename}
Called upon completion of parsing of an included file.  The filename
that what just finished being parsed is the only argument.
\hook{TOKEN}{\$token, \$raw, \$macro\_\-name, \$arg\_\-num, @history}
Called once for each token CPP reads.  \$token is the type of token,
with a leading CPP\_\-; \$raw is the raw characters which constitute that
token; \$macro\_\-name is the name of the macro that expanded to create this token,
or is empty if the token did not result from macro expansion; \$arg\_\-num is -1
if the token did not come from a macro expansion, 0 if it came from the
body of a macro expansion (i.e., not an argument), or positive and equal
to the argument number that produced the token, if from an substitution
of an argument in a macro expansion; @history is obsoleted, use the
MacroExpansionHistory() backcall instead.  Note that the token "CPP\_\-POP"
means that an expansion buffer has been popped from the stack of expansions,
and is not a real lexical token.  This may be removed from future version
so use the POP\_\-BUFFER hook to handle those events.
\hook{FUNCTION}{\$name, \$fStatic}
Called once after an entire function is parsed.  Arguments
give the name of the newly defined function, and \$fStatic is non-zero
iff the function was declared to be static (i.e. not global).
\hook{FUNC\_\-PROTO}{\$name}
Called once after an function prototype (declaration) is parsed.  Argument
gives the name of the newly defined function.  This hook is not called
for complete function definitions -- see HI\_\-FUNCTION for that hook.
\hook{FUNC\_\-CALL}{\$name}
Called once after each function call is parsed.  Only argument is
the name of the function being called.
\hook{TYPEDEF}{\$name}
Called once after each typedef is parsed.  Argument is the name
of the new type.
\hook{VARDECL}{\$name}
Called once after each variable declaration is parsed.  Argument is the name
of the declared variable.
\hook{ANNOTATE}{} Obsoleted
\hook{POP\_\-BUFFER}{\$cbuffersDeep}
Called once for each buffer that is popped off of the stack of
buffers to be parsed. The only argument is the new (after the pop)
number of non-file buffers
deep the stack is.  In particular, when that number is 0, 
the parser is not parsing a macro expansion any longer.
\hook{POP\_\-PERL\_\-BUFFER}{\$cbuffersDeep}
Called once for each perl-pushed buffer that is popped off of the stack of
buffers to be parsed (i.e., those pushed with the PushBuffer() backcall)
The only argument is the new (after the pop)
number of non-file buffers
deep the stack is.  In particular, when that number is 0, 
the parser is not parsing a macro expansion any longer.

\fussy
\end{itemize}
\end{footnotesize}

\newpage
\section{Subroutines provided by \pcp}
\label{sec:backcalls}
\begin{footnotesize}
\begin{itemize}
\sloppy
\backcall{\$token\_\-type}{SzToken}{\$id}
Returns a string identifying what kind of token \textit{id} (an integer) represents.
\backcallobsoleted{(\$cchOffset,\$cbuffersback)}{SumCchExpansionOffset}{}
Gives the source code character position offset and the number of expansions
deep that we currently are.
\backcall{\$cbuffersback}{CbuffersBack}{}
Return the number of macro expansions deep that we currently are in
the current expansion.  This returns 0 if we are not expanding a macro.
The number returned is the number of non-file buffers in the current
stack of expansions (so it is not increased by nested \pphash{}include-s).
\backcall{@expanded\_\-macros\_\-list}{MacroExpansionHistory}{}
Returns a list of strings of the form 
\textit{macro\_\-name}\pphash{}\textit{arg\_\-num}[\textit{cchOffsetEnd}]. 
Each string explains the contents of a current expansion buffer,
and the first string is the top of the expansion stack.
For example, if there is a single item in the returned list
which is the string "MAX\pphash{}1[3]", this means that the current
token came from the expansion of macro MAX and was the first argument
to that macro, ending at character offset 3.  Multiple items
in the returned list mean that a macro expanded to some other
macro which was subsequently expanded.  Argument number 0 signifies
that the token came from the literal body of the macro.
\backcallobsoleted{\$index\_\-of\_\-argument}{ArgOf}{}
Returns the number of the argument from which the current token came.
Prefer using the \$argno parameter passed to the TOKEN hook.
\backcall{\$cch\_\-source\_\-offset}{CchOffset}{}
Returns the character position offset into the current source file.
Use Fname() to get the source file name.
\backcall{\$filename}{InFname}{}
Returns a string giving the name of the main input file (the one
that appeared on the command line.
\backcall{\$filename}{Fname}{}
Returns a string giving the name of the current input file.  This
tracks include-d files and, combined with CchOffset, gives a
unique location in the package.  "@NONE@" returned if no current file.
\backcall{\$filename}{FnameNominal}{}
Returns a string giving the filename corresponding to the current
buffer.  This will be "@NOFILE@" if there is no such file (as is
the case when reading tokens from a macro expansion).
Otherwise it will correspond to Fname().
\backcall{\$encoded\_\-expansion}{ExpansionLookup}{\$macro\_\-name}
Returns the internally-encoded (cpplib-specific) expansion of
the macro with name \texttt{\$macro\_\-name}.  Note that this the string
returned is delimited on either end by "@ " and omits argument occurrences.
"@NOTFOUND@" is return if there is no current macro of the given name.
\backcall{\$cchOutput}{CchOutput}{}
Returns the number of characters already output.  Only non-zero when 
--noparse option is given.
\backcallobsoleted{\$cchCppRead}{CchCppRead}{}
Returns the number of characters that have been read from the
input files.  Prefer using the offsets passed in the various hooks.
\backcall{\$fExpandingMacros}{FExpandingMacros}{}
Returns TRUE iff macros are currently being expanded.
Returns FALSE otherwise.  As arguments are read, macros are not expanded,
and this will return FALSE.
\backcall{@parse\_\-state\_\-list}{ParseStateStack}{}
Returns the current stack of parse state numbers.  The first element is
the top state on the stack.  See the gram.output
file for a listing of what the numbers correspond to.
\backcall{}{SetParseStateStack}{@parse\_\-state\_\-list}
Resets the current stack of parse states so that it is @parse\_\-state\_\-list.
This has the potential to break the parse, and often will outside
of the trivial case of setting the stack of states to what it already is.
\backcall{}{PushBuffer}{\$source\_\-code\_\-string, \$s\_\-start}
Adds \$source\_\-code\_\-string to the being-processed text 
as if it existed in the input at the
current location in the current file. \$s\_\-start tells
where the code to be processed is in the current source
file, or is negative if it does not exist there.
\backcall{}{EnterScope}{}
Enters a new scoping level; useful in combination
with PushBuffer() for causing code to be parsed without
affecting the current symbol table directly.
\backcall{}{ExitScope}{}
Exits the current scoping level; useful in combination
with PushBuffer() for causing code to be parsed without
affecting the current symbol table directly.
\backcall{}{PushHashTab}{}
Copies the current cpp hash table and begins using the copy;  the copy
can later be thrown out using PopHashTab()
\backcall{}{PopHashTab}{}
Throw out a previously pushed cpp hash table; revert to the prior
hash table
\backcall{}{SetParseDebugging}{}
Set the yydebug flag to TRUE for the underlying parser.
This results in extra state change information being sent to stderr.
It has no affect if --noparse is used.  Initially, parser debugging is off.
\backcall{}{ResetParseDebugging}{}
Reset the yydebug flag to FALSE for the underlying parser.  
This terminates sending extra state change information to stderr.
It has no affect if --noparse is used.
\backcall{}{YYPushStackState}{}
Save the entire current state of the parse stack onto a meta stack.
\backcall{}{YYPopAndRestoreStackState}{}
Restore the entire current state of the parse stack from the meta stack.
\backcall{}{YYPopAndRestoreStackState}{}
Throws out the top state stack from the meta stack.
\backcall{}{YYSwapStackState}{}
Switch the current stack state and the one on the top of the meta stack.
\backcall{}{YYPushDupTopStackState}{}
Push another copy of the top element of the meta stack of stack states onto
the meta stack.
\backcall{\$fStacksEqual}{YYFCompareTopStackState}{}
Return TRUE iff the top of the meta stack of state stacks is
identical to the current state stack. Return FALSE otherwise.
\backcall{\$fDefined}{FLookupSymbol}{\$symbol\_\-id}
Return TRUE iff \$symbol\_\-id is found in the current scope.
Return FALSE otherwise.

\fussy
\end{itemize}
\end{footnotesize}

% References
\newpage

\nocite{ARM}
\nocite{Dragon}
\nocite{Glickstein97} % Writing GNU Emacs Ext.
\nocite{Camel}        % Perl 5
\nocite{Levine92}     % Lex & Yacc
\nocite{Harbison91}   % C Ref Man
\nocite{Stroustrup97} % C++, 3rd
\nocite{Stroustrup94} % C++, 2nd
\nocite{Kernighan88}  % C, 2nd
\nocite{Flanagan96}   % Java in a Nutshell
\nocite{BtYACC}
\nocite{EmpCpp-TR}
\nocite{GCC}
\nocite{CTree}
\nocite{TXL}
\nocite{Cordy92}
\nocite{BtYACC}
\nocite{Bison}
\nocite{Flex}
\nocite{Krone94}
\nocite{Griswold96}
\nocite{Atkinson96}
\nocite{CD2DraftStandard}

\bibliographystyle{alpha}
\bibliography{library,articles,pcp3}

\end{document}

%%% Local Variables: 
%%% mode: latex
%%% TeX-master: t
%%% End: 
